%\documentclass{article}
%\usepackage[
%    left=1.2in,
%    right=1.2in,
%    top=0.4in,
%    bottom=0.7in,
%    paperheight=11in,
%    paperwidth=8.5in
%]{geometry}

%\usepackage{layout}
\clearpage
\mytitle{Communication-Efficient Secure Logistic Regression \large \\Paper Summary}
%\renewcommand{\title}{TVA: A multi-party computation system for secure and expressive time series analytics \large \\Paper Summary}
%\author{Sandhya Saravanan\\
%  \small MSR, India\\\\
%}
%\date{\vspace{-5ex}}

%\begin{document}
%\maketitle
\setcounter{section}{0} % Restart section numbering

\section{Summary}
Spline (2 rounds, communicates 6 elements) $\rightarrow$ Communication independent of no. of intervals

\subsection{Technical Contributions}
\begin{enumerate}
    \item New Approximation technique for Sigmoid
    \begin{itemize}
        \item For input interval [0, 1), we use spline approximation that splits the interval into several pieces, each of which is approximated with a linear function.
        \item For interval between 1 and a configurable threshold, we use Taylor approximation.
        \item For large values above threshold, we approximate sigmoid with value 1
        \item Negative inputs are handled symmetrically.
        \item Multiple Interval Containment (MIC) gate (from the Eurocrypt paper) is used to identify which interval the input falls into in order to use the approximation function. MIC gate is also used within the spline approximation on the interval [0, 1) to choose the right linear function.
        \item MIC gate is optimized by introducing a reduction from DCF to IDPF.
    \end{itemize}
    \item Secure spline evaluation and secure powers computation protocols for fixed point values
    \item New constant round comparison protocol
        \begin{itemize}
            \item For comparison of $l$-bit nos, total 3 communication rounds while state-of-art SynCirc uses rounds logarithmic in $l$
        \end{itemize}
    \item New 2-party protocol to generate keys for DPFs over arithmetic sharing (previous constructions only do this for Boolean outputs)
        \begin{itemize}
            \item Doerner Shelat ORAM only handled generating DPF keys for Boolean shared outputs. This work provides a construction to generate DPF keys for arithmetic shared outputs too. This requires a single additional OT in offline phase, independent of size of output shares.
        \end{itemize}
\end{enumerate}

\subsection{Setting}
2PC. Each party holds a cryptographic share of input data. 

\subsection{Secure Spline Computation}
A spline function $S : \mathbb{R} \rightarrow \mathbb{R}$ on an interval [a, b) is specified as a partition of $m$ intervals $\{a_i, b_i\}_{i \in m}$ with a d-degree polynomial $p_i$ defined for each of the intervals. The value of the function $S$ on input $x \in [a, b]$ is equal to $p_i(x)$ where $a \le x < b$.\\

For the case of Sigmoid, we use degree 1 polynomials on $m$ intervals. Note that such a polynomial $Q(x)$ is of the form $Q(x) = ax + b$ where $a, b$ are publicly known values. Given a secret sharing of x, parties can locally compute a sharing of $Q(x)$. \\

\underline{Spline protocol}
\begin{enumerate}
    \item Let the 2 parties locally evaluate degree 1 polynomials $Q_i$ defined for each of the m intervals.
    \item Let $\overline{Q}$ represent a length m vector containing the result of evaluating $Q_i$ on $x$ for each of the $m$ intervals.
    \item Parties use MIC gate to generate shares of a vector $\overline{B} = [b_1, b_2, ..., b_m]$ where $b_i = 1\{p_i \le x < q_i\}$.
    \item They compute secure dot product of $\overline{Q}$ and $\overline{B}$ to derive actual spline result.
    \item Total communication: 2 + 4m elements, Online rounds: 2 (1st round for MIC gate evaluation, 2nd round for matrix multiplication)
    \item This work provides optimized protocol for performing dot product (for splines specifically) which reduces total communication to 6 elements. Note that the online cost is independent of no. of intervals i.e. $m$.
\end{enumerate}

% Bibliography
%-----------------------------------------------------------------
%\begin{thebibliography}{99}

%\bibitem{HamCCS23} Gilad Asharov and Koki Hamada and Dai Ikarashi and Ryo Kikuchi and Ariel Nof and Benny Pinkas and Junichi Tomida \emph{Secure Statistical Analysis on Multiple Datasets: Join and Group-By}, {CCS} (2023)

%\end{thebibliography}

%\end{document}
