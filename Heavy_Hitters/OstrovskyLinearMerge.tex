%\documentclass{article}
%\usepackage[
%    left=1.2in,
%    right=1.2in,
%    top=0.4in,
%    bottom=0.7in,
%    paperheight=11in,
%    paperwidth=8.5in
%]{geometry}

%\usepackage{layout}
\clearpage
\mytitle{Linear-Time Secure Merge in O(log log n) Rounds \large \\Paper Summary}
%\renewcommand{\title}{TVA: A multi-party computation system for secure and expressive time series analytics \large \\Paper Summary}
%\author{Sandhya Saravanan\\
%  \small MSR, India\\\\
%}
%\date{\vspace{-5ex}}

%\begin{document}
%\maketitle
\setcounter{section}{0} % Restart section numbering

\section{Summary}
\underline{Secure Merge Problem}: Given 2 secret shared sorted lists of size m and n resp, combine them into a single secret shared sorted list.\\

\underline{Contributions}:
\begin{enumerate}
    \item Semi-honest security
    \item Results based solely on black box use of basic secure primitives.
    \item Computationally secure 2-party merge (since 2-party secure primitives requires computational assumptions)
    \item Information-theoretically 3-party merge (since 3-party secure primitives do not require computational assumptions)
    \item They make the following conjecture and claim that their 2P secure merge protocol is optimal
    \begin{itemize}
        \item Any linear time protocol for 2 parties to securely merge their lists (each of size $\Theta (n)$ requires $\Omega (\log \log n)$ rounds of communication
    \end{itemize}
    \item \underline{Base idea}: Split each list into blocks of size $\sqrt n$ with $\sqrt n$ medians.
    \begin{itemize}
        \item Round 1: Run pairwise comparisons between medians to identify which block of the opposite list each median is mapped to
        \item Round 2: Run pairwise comparisons to identify exact position the median is mapped to within that block.
        \item Result: $\sqrt n$ subproblems of size $\sqrt n$
    \end{itemize}
    \item For $k > 3$ parties, merging k lists of size $O(n)$ $\rightarrow$ $O(\log \log n + k)$ round complexity
    \item \textbf{Table 1 that contains comparison of various secure merge protocols is important}. In this table, parameters $\beta_{HE}$ and $\gamma$ arise out of use of homomorphic encryption for two party shuffle, therefore relevant for comparing secure merge protocols against 2P merge networks with standard MPC. They can be set to $1$ in 3P setting.
    \begin{itemize}
        \item Secure $(n^\alpha , n)$ Merge $\rightarrow$ Computation: $O(2^{1/(1-\alpha)^3} \gamma n)$ Communication: $O(2^{1/(1-\alpha)^3} \beta_{HE} n)$ Rounds: $O(1/(1-\alpha)^3)$
        \item Secure $(n^{1/3} , n)$ Merge $\rightarrow$ Computation: $O(n)$ Communication: $O(n)$ Rounds: $O(1)$
        \item Secure $(n , n)$ Merge $\rightarrow$ Computation: $O(\gamma n)$ Communication: $O(\beta_{HE} n)$ Rounds: $O(\log \log n)$

    \end{itemize}
    \item \underline{Secure Merge Strategy}: Partition original lists into several blocks, then align these blocks (find the appropriate block(s) from the other list that span the "same" range of values); perform secure merge on these blocks; and finally combine (concatenate) the blocks together to obtain the final merged list.
    \begin{itemize}
        \item Imagine we partition to form k blocks
        \item Each block has $n/k$ elements
        \item For $k = n/\log \log n$, apply the linear communication and linear round secure merge to each block
        \item Since each block contains $n/k = \log \log n$ elements, each subproblem requires $O(\log \log n)$ communication, computation and round-complexity.
        \item Since each subproblem can be performed in parallel, total cost across all $k = n/\log \log n$ blocks will be $O(n)$ computation and communication but still only $O(\log \log n)$ rounds.
    \end{itemize}
    \item Additive secret sharing
    \item Length of list elements treated as constant
\end{enumerate}

% Bibliography
%-----------------------------------------------------------------
%\begin{thebibliography}{99}

%\bibitem{TVA22} Muhammad Faisal, Boston University; Jerry Zhang \emph{TVA: A multi-party computation system for secure and expressive time series analytics}, {USENIX} (2023)

%\end{thebibliography}

%\end{document}
