%\documentclass{article}
%\usepackage[
%    left=1.2in,
%    right=1.2in,
%    top=0.4in,
%    bottom=0.7in,
%    paperheight=11in,
%    paperwidth=8.5in
%]{geometry}

%\usepackage{layout}
\clearpage
\mytitle{Efficient Secure Two-Party Computation using Symmetric Cut-and-Choose \large \\Paper Summary}
%\renewcommand{\title}{TVA: A multi-party computation system for secure and expressive time series analytics \large \\Paper Summary}
%\author{Sandhya Saravanan\\
%  \small MSR, India\\\\
%}
%\date{\vspace{-5ex}}

%\begin{document}
%\maketitle
\setcounter{section}{0} % Restart section numbering

\section{Summary}

\underline{Symmetric cut-and-choose}: Both parties generate $\kappa$ circuits to be checked by the other party. 

\subsection{Previous Techniques}
Using $\kappa$ GCs yields security level $2^{-0.32\kappa}$

\subsection{Selling Point}
\begin{enumerate}
    \item $\kappa$ reduced by $3 \times$ with same statistical security level 
    \item $3 \times$ running time improvement over existing schemes 
\end{enumerate}

\section{Technique}
\begin{enumerate}
    \item Each party generates $\kappa$ GCs to be checked by the other party.
    \item After checking, each party evaluates remaining GCs
    \item A party outputs value $v$ for some output wire of circuit iff at least one of their own GCs, and at least one of the GCs generated by the other party, evaluate to $v$ on that wire.
\end{enumerate}

\subsection{Correctness}
There is at least one honest party. An honest party always generates correct GCs. Let it output $v$. If at least one of the other party's generated GCs also outputs $v$, then $v$ is the correct output.

\subsection{Security}
Consider a malicious party. Let $c$ be the no. of circuits checked. The malicious party cheats if it generates $\kappa-c$ bad GCs, and none of those is checked by the other party. Probability of successful cheating = $1/\binom{\kappa}{c} = 2^{-\kappa+O(\log \kappa)}$ which is lowest for $c = \kappa/2$

Naor-Pinkas OT used to ensure input consistency.
% Bibliography
%-----------------------------------------------------------------
\begin{thebibliography}{99}

\bibitem{LP07} Yehuda Lindell, Benny Pinkas \emph{An efficient protocol for secure two-party computation in the presence of malicious adversaries}, {EUROCRYPT} (2007)
\bibitem{LP11} Yehuda Lindell, Benny Pinkas \emph{Secure two-party computation via cut-and-choose oblivious transfer}, {TCC} (2011)
\bibitem{sS11} Abhi Shelat and Chih-Hao Shen \emph{Two-output secure computation with malicious adversaries}, {EUROCRYPT} (2011)
\bibitem{Lin13} Yehuda Lindell, Bar-Ilan University \emph{Fast Cut-and-Choose Based Protocols for Malicious and Covert Adversaries}, {CRYPTO} (2013)
\bibitem{Bra13} Luis T. A. N. Brandao \emph{Secure two-party computation with reusable bit-commitments, via a cut-and-choose with forge-and-lose technique}, {ASIACRYPT} (2013)
\bibitem{HKE13} Yan Huang, Jonathan Katz, and David Evans \emph{Efficient secure two-party computation using symmetric cut-and-choose}, {CRYPTO} (2013)
\bibitem{LR14} Yehuda Lindell, Ben Riva \emph{Cut-and-choose Yao-based secure computation in the online/offline and batch settings}, {CRYPTO} (2014)
\bibitem{HKK+14} Yan Huang, Jonathan Katz, Vladimir Kolesnikov, Ranjit Kumaresan and Alex J Malozemoff \emph{Amortizing garbled circuits}, {CRYPTO} (2014)
\bibitem{RR16} Peter Rindal, Mike Rosulek \emph{Faster malicious 2-party secure computation with online/offline dual execution}, {EUROCRYPT} (2017)
\bibitem{WMK17} Xiao Wang, Alex J. Malozemoff, Jonathan Katz \emph{Faster secure two-party computation in the single-execution setting}, {EUROCRYPT} (2017)
\bibitem{NST17} Jesper Buus Nielsen, Thomas Schneider, and Roberto Triletti \emph{Constant round maliciously
secure 2PC with function-independent preprocessing using LEGO}, {NDSS} (2017)

\end{thebibliography}

%\end{document}
