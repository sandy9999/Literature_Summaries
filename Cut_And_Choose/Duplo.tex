%\documentclass{article}
%\usepackage[
%    left=1.2in,
%    right=1.2in,
%    top=0.4in,
%    bottom=0.7in,
%    paperheight=11in,
%    paperwidth=8.5in
%]{geometry}

%\usepackage{layout}
\clearpage
\mytitle{DUPLO: Unifying Cut-and-Choose for Garbled Circuits \large \\Paper Summary}
%\renewcommand{\title}{TVA: A multi-party computation system for secure and expressive time series analytics \large \\Paper Summary}
%\author{Sandhya Saravanan\\
%  \small MSR, India\\\\
%}
%\date{\vspace{-5ex}}

%\begin{document}
%\maketitle
\setcounter{section}{0} % Restart section numbering

\section{Summary}

\subsection{Key Idea}
Traditional cut-and-choose operates at the level of entire circuits. In LEGO, cut-and-choose is at the level of individual gates. In this work, cut-and-choose operates on the level of arbitrary circuit "components" which can range from single gate to entire circuit.

\subsection{Contributions}
\begin{enumerate}
    \item Previous protocol techniques modified and optimized to work with general Cut-And-choose components 
    \item Extension of Frigate circuit compiler to effectively express any C-style program in terms of components that can be efficiently processed using this protocol. 
\end{enumerate}

\subsection{Previous Techniques}
Typical ways to make malicious security more practical. Improvements seen in:
\begin{enumerate}
    \item Reduce no. of GCs 
    \item Trade-offs between offline and online computation phases
    \item Slight weakenings of security
\end{enumerate}

\underline{Replication Factor}: Multiplicative overhead in garbling material due to replication (of circuits in cut-and-choose).

\subsection{Replication Factor in Previous Works}
For statistical security $2^{-s}$ \\

Cut-And-Choose
\begin{enumerate}
    \item \cite{sS11}: $3s$ (Evaluation phase requires majority of unopened circuits to be correct)
    \item \cite{Bra13} \cite{Lin13}: $s$ (Evaluation phase requires at least 1 unopened circuit to be correct) 
    \item \cite{LR14} \cite{HKK+14}: $O(1) + O(s/\log N)$ (Amortized setting where parties perform $N$ independent evaluations of same circuit. So all evaluations share common cut-and-choose phase where only small fraction of circuits to be opened)
    \item LEGO: $2 + O(s/\log N)$ (If Evaluated function has $N$ gates)
\end{enumerate}

\subsection{Selling Point}
\begin{enumerate}
    \item $4-7 \times$ running time improvement over \cite{WMK17} for certain circuits
    \item $4 \times$ running time improvement over \cite{RR16} in multi-execution setting
\end{enumerate}

% Bibliography
%-----------------------------------------------------------------
\begin{thebibliography}{99}

\bibitem{LP07} Yehuda Lindell, Benny Pinkas \emph{An efficient protocol for secure two-party computation in the presence of malicious adversaries}, {EUROCRYPT} (2007)
\bibitem{LP11} Yehuda Lindell, Benny Pinkas \emph{Secure two-party computation via cut-and-choose oblivious transfer}, {TCC} (2011)
\bibitem{sS11} Abhi Shelat and Chih-Hao Shen \emph{Two-output secure computation with malicious adversaries}, {EUROCRYPT} (2011)
\bibitem{Lin13} Yehuda Lindell, Bar-Ilan University \emph{Fast Cut-and-Choose Based Protocols for Malicious and Covert Adversaries}, {CRYPTO} (2013)
\bibitem{Bra13} Luis T. A. N. Brandao \emph{Secure two-party computation with reusable bit-commitments, via a cut-and-choose with forge-and-lose technique}, {ASIACRYPT} (2013)
\bibitem{HKE13} Yan Huang, Jonathan Katz, and David Evans \emph{Efficient secure two-party computation using symmetric cut-and-choose}, {CRYPTO} (2013)
\bibitem{LR14} Yehuda Lindell, Ben Riva \emph{Cut-and-choose Yao-based secure computation in the online/offline and batch settings}, {CRYPTO} (2014)
\bibitem{HKK+14} Yan Huang, Jonathan Katz, Vladimir Kolesnikov, Ranjit Kumaresan and Alex J Malozemoff \emph{Amortizing garbled circuits}, {CRYPTO} (2014)
\bibitem{RR16} Peter Rindal, Mike Rosulek \emph{Faster malicious 2-party secure computation with online/offline dual execution}, {EUROCRYPT} (2017)
\bibitem{WMK17} Xiao Wang, Alex J. Malozemoff, Jonathan Katz \emph{Faster secure two-party computation in the single-execution setting}, {EUROCRYPT} (2017)
\bibitem{NST17} Jesper Buus Nielsen, Thomas Schneider, and Roberto Triletti \emph{Constant round maliciously
secure 2PC with function-independent preprocessing using LEGO}, {NDSS} (2017)

\end{thebibliography}

%\end{document}
