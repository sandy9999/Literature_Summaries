%\documentclass{article}
%\usepackage[
%    left=1.2in,
%    right=1.2in,
%    top=0.4in,
%    bottom=0.7in,
%    paperheight=11in,
%    paperwidth=8.5in
%]{geometry}

%\usepackage{layout}
\clearpage
\mytitle{Secure Two-Party Computation via Cut-and-Choose Oblivious Transfer \large \\Paper Summary}
%\renewcommand{\title}{TVA: A multi-party computation system for secure and expressive time series analytics \large \\Paper Summary}
%\author{Sandhya Saravanan\\
%  \small MSR, India\\\\
%}
%\date{\vspace{-5ex}}

%\begin{document}
%\maketitle
\setcounter{section}{0} % Restart section numbering

\section{Summary}

\subsection{Motivation}
A malicious adversary who constructs the garbled circuit may construct a garbling of a different circuit computing a different function, and this cannot be detected (due to garbling).

\subsection{Solution Methodology: Cut and Choose}
Many circuits sent, some of them are opened to check that they are correct while others are evaluated. Known to be the best methodology for Boolean circuits.

\subsection{Selling Point}
\begin{enumerate}
    \item For cheating probability of $2^{-40}$, best previous works send between 680 circuits. In this work, 132 circuits suffice.
    \item Previous works achieve proven cheating probability of $2^{-s/17}$ and conjectured probability of $2^{-s/4}$. This work achieves cheating probability of $2^{-0.311s}$ where $s$ = No. of garbled circuits.
    \item Previous works' cost
    \begin{itemize}
        \item For input length $l$, complexity: $O(l)$ OTs, construction and computation of $s$ circuits, $2s^2l$ commitments
    \end{itemize}
    \item This works' cost
    \begin{itemize}
        \item $5.66sl$ full exponentiations
        \item Commn: $7sl + 22l + 7s + 5$ group elements
        \item 12 rounds commn
        \item $6.5|C|s$ symmetric encryptions (constructing and decrypting GCs)
        \item $4|C|s$ ciphertexts (to transmit these GCs)
    \end{itemize}
    \item Moderate increase in no. of exponentiations, reduced no. of circuits, commitments completely removed, no need to increase size of inputs.
\end{enumerate}

\subsection{Previous Works' Protocol}
\begin{enumerate}
    \item $P_1$ constructs a large no. of garbled circuits and sends them to $P_2$.
    \item $P_2$ chooses a subset of circuits which are opened and checked.
    \item If all these circuits are correct, then remaining circuits are evaluated as in Yao's protocol, and $P_2$ outputs majority output value.
\end{enumerate}

This work relies on decisional DDH assumption, while previous works relied on general assumptions. \\

\underline{Decisional DDH Assumption}: Given some group $G$ and generator $g$, it is "hard" to distinguish between the distributions: $\langle g^a, g^b, g^{ab}\rangle$ and $\langle g^a, g^b, g^c\rangle$.

\subsection{Other approaches}

\subsubsection{Approach 1: ZK Proving/Committed Input Method}
\begin{enumerate}
    \item Construct a single circuit, and prove that it's correct (E.g. used correct keys, gate has correct structure etc)
    \item Cost: $O(1)$ exponentiations/gate (Here $O(1)$ is 720), more expensive than DH exponentiations in EC group.
    \item Scope for optimization:
    \begin{itemize}
        \item How to build gates so that they yield efficient proofs, and therefore, more efficient ZK protocols?
        \item Batching of ZK protocols?
    \end{itemize}
\end{enumerate}

\underline{Fault Tolerant Circuit}: A circuit that can tolerate some no. of incorrect gates and still correctly compute the function. 
\subsubsection{Approach 2: LEGO}
\begin{enumerate}
    \item Circuit constructor first sends the receiver many gates (encrypted using homomorphic (Petersen) commitments)
    \item Receiver asks to open half of the gates to check that they are correctly formed (Guarantees that majority of remaining gates are correct). 
    \item Parties interact in such a way that the remaining gates are securely soldered into a fault tolerant circuit (using homomorphic property)
    \item Size of fault tolerant circuit: $O(s|C|/\log|C|)$. $s$ is statistical security param. Error is $2^{-s}$.
    \item Why is the size of the fault tolerant circuit so big? For every gate to be computed, we replace it with $s$ gates and take the majority result of these gates.
    \item No. of exponentiations (DH) / gate = 32
\end{enumerate}

\subsubsection{Approach 3: Virtual Multiparty Method}
\begin{enumerate}
    \item Parties simulate a virtual multiparty protocol with an honest majority. 
    \item Cost of protocol = Cost of running semi-honest protocol for computing additive shares of the product of additive shares, for every multiplication carried out by a party in a multiparty protocol with honest majority (didn't get this??) 
    \item Efficiency = f(Multiparty protocol to be simulated, semi-honest protocol to simulate the multiparty protocol) 
\end{enumerate}

\subsubsection{Approach 4: Multiplication Tuples (Arithmetic Circuits)}
\begin{enumerate}
    \item Parties share their inputs like in BGW
    \item Assume that parties have many tuples of the form [Com(x), Com(y), Com(z)] where $x = y.z$ together with additive shares $(x_1, x_2), (y_1, y_2), (z_1, z_2)$ of $x, y, z$. 
    \item Note: Com is homomorphic
    \begin{itemize}
        \item Given share of Com(x), Com(y), we can compute shares of Com(x+y).
        \item Given shares of a, x, we can compute shares of Com(a.x). 
    \end{itemize}
    \item Setting 
    \begin{itemize}
        \item Wire 1: $P_1$ and $P_2$ have additive shares $u_1$, $u_2$ of $u$.
        \item Wire 2: $P_1$ and $P_2$ have additive shares $v_1$, $v_2$ of $v$.
        \item Aim: Compute shares of $w=u.v$ i.e. compute $w_1$, $w_2$ s.t. $w_1 + w_2 = (u_1 + u_2).(v_1 + v_2)$
        \item $P_1$ and $P_2$ have additive shares of Com(x), Com(y), Com(z) where $x =y.z$.
        \item They compute shares of Com(u-y) and open. Let's call this u'
        \item They compute shares of Com(v-z) and open. Let's call this v'
        \item They compute shares of Com(u'.v) + Com(v'.u) + Com(x) - u'.v'
    \end{itemize}
\end{enumerate}

%\subsection{Approach of this Paper: Cut and Choose}
%Considered the best for boolearn circuits ? \\
%\begin{enumerate}
%    \item Garbler ($P_1$) constructs $s$ circuits instead of 1 circuit and sends them all to evaluator ($P_2$).
%    \item $P_2$ chooses a random subset of half of those circuits and asks $P_1$ to open them i.e. $P_2$ gets access to all the keys corresponding to the input wires.
%    \item $P_2$ decrypts the entire circuit and checks that it really is a circuit computing the function.
%    \item Guarantee: Majority of the remaining circuits are correctly constructed
%    \item $P_1$ and $P_2$ compute all of the remaining garbled circuits (computing 1 not enough as it is only guaranteed that the majority are correctly constructed)
%    \item What does $P_2$ do if it obtains different outputs?
%    \begin{itemize}
%        \item What if $P_2$ aborts? $P_1$ can execute the following attack. $P_1$ constructs 1 circuit that outputs garbage if the first bit of $P_2$'s input equals 0 (otherwise, computes $f$). If $P_2$ aborts, $P_1$ knows that $P_2$'s 1st input equals 0.
%        \item What if $P_2$ outputs the majority value? Correct option. 
%    \end{itemize}
%    \item What if $P_1$ constructs a garbled circuit $G$ with 2 different sets of garbled values / keys $K$, $K'$ s.t. the keys in $K$ decrypt $G$ to the correct circuit $C$ and the keys in $K'$ decrypt $G$ to an incorrect circuit $C'$. This can be solved by having $P_1$ commit to the input keys. Later on, the computation happens on committed keys, not on arbitrary keys.
%    \item Input consistency: What if $P_2$ uses different inputs $y_1$, $y_2$, ... in different circuits to get $f(x, y_1)$, $f(x, y_2)$ etc. Similarly, what if $P_1$ uses different inputs $x_1$, $x_2$, ... in different circuits to get $f(x_1, y)$, $f(x_2, y)$ etc. $P_2$ may not detect this from the output, as the function can be customized accordingly such that it doesn't.
%    \item Cut and Choose doesn't help with selective input attack. Solution to prevent selective input attack: 
%    \begin{itemize}
%        \item Split each input bit $y$ of $P_2$ into $s$ random bits $y_1, ..., y_s$ such that $y_1 \oplus y_2 \oplus ... \oplus y_s = y$
%        \item Change circuit to first compute XOR of these bits and then the function.
%        \item Each input bit is now random (correlation between $y_1$, ..., $y_s$ and actual bit $y$ can be guessed with probability $2^{-s}$). Thus, any attack on input bits is not correlated to actual input.
%    \end{itemize}
%
%\end{enumerate}

\section{Techniques}
\subsection{Consistency of $P_1$'s inputs}
$P_1$ determines keys associated with its own input bits by choosing values $g^{a_1^0}$, $g^{a_1^1}$, ..., $g^{a_l^0}$, $g^{a_l^1}$ and $g^{r_1}$, ..., $g^{r_s}$ and then sets keys associated with i-th input bit in j-th circuit to be $g^{a_i^0.r_j}$, $g^{a_i^1.r_j}$. Given $g^{a_i^0}$, $g^{a_i^1}$, $g^{r_j}$ and any subset of keys of $P_1$'s input generated this way, remaining keys associated with its input are pseudorandom by DDH assumption. Garbled values for rest of the circuit are chosen as usual (?).

\subsection{Primitive: Cut-and-Choose OT}
1-out-of-2 OT s.t. in some of the transfers ($s/2$ transfers in this work), receiver obtains single value (like regular OT), while in others, obtains both values. \\

\underline{Use Case in Yao's GC}: Receiver obtains all garbled input values in circuits to be \underline{checked}, and only garbled input values associated with input on circuits to be \underline{evaluated}. \\

In this work, implemented under DDH assumption.

%\subsection{Approach 1: ZK Proving}
%\begin{enumerate}
%    \item Encrypt gates using asymmetric encryption with algebraic structure. Use Camenisch-Shoup based on DCR (N-residuosity): 2 exponentiations mod $N^2$
%    \item Use structure to prove in ZK that circuit is correctly constructed. E.g. used correct keys, gate has correct structure etc
%    \item Cost: $O(1)$ exponentiations/gate (Here $O(1)$ is 720), more expensive than DH exponentiations in EC group.
%    \item Scope for optimization:
%    \begin{itemize}
%        \item How to build gates so that they yield efficient proofs, and therefore, more efficient ZK protocols?
%        \item Batching of ZK protocols?
%    \end{itemize}
%\end{enumerate}
%
%\subsection{Approach 2: LEGO}
%\begin{enumerate}
%    \item Generate many encrypted gates using homomorphic (Petersen) commitments
%    \item Open half of the gates to check that they are correctly formed (Guarantees that majority of remaining gates are correct). 
%    \item Combine remaining gates in fault tolerant circuit (can tolerate some no. of incorrect gates and still correctly compute the function). Use homomorphic property to solder gates
%    \item Compute above circuit 
%    \item Size of fault tolerant circuit: $O(s|C|/\log|C|)$. $s$ is statistical security param. Error is $2^{-s}$.
%    \item Why is the size of the fault tolerant circuit so big? For every gate to be computed, we replace it with $s$ gates and take the majority result of these gates.
%    \item No. of exponentiations (DH) / gate = 32
%\end{enumerate}
%
%\subsection{Approach 4: Multiplication Tuples (Arithmetic Circuits)}
%\begin{enumerate}
%    \item Parties share their inputs like in BGW
%    \item Assume that parties have many tuples of the form [Com(x), Com(y), Com(z)] where $x = y.z$ together with additive shares $(x_1, x_2), (y_1, y_2), (z_1, z_2)$ of $x, y, z$. 
%    \item Note: Com is homomorphic
%    \begin{itemize}
%        \item Given share of Com(x), Com(y), we can compute shares of Com(x+y).
%        \item Given shares of a, x, we can compute shares of Com(a.x). 
%    \end{itemize}
%    \item Setting 
%    \begin{itemize}
%        \item Wire 1: $P_1$ and $P_2$ have additive shares $u_1$, $u_2$ of $u$.
%        \item Wire 2: $P_1$ and $P_2$ have additive shares $v_1$, $v_2$ of $v$.
%        \item Aim: Compute shares of $w=u.v$ i.e. compute $w_1$, $w_2$ s.t. $w_1 + w_2 = (u_1 + u_2).(v_1 + v_2)$
%        \item $P_1$ and $P_2$ have additive shares of Com(x), Com(y), Com(z) where $x =y.z$.
%        \item They compute shares of Com(u-y) and open. Let's call this u'
%        \item They compute shares of Com(v-z) and open. Let's call this v'
%        \item They compute shares of Com(u'.v) + Com(v'.u) + Com(x) - u'.v'
%    \end{itemize}
%\end{enumerate}
%
%\subsection{Approach of this Paper: Cut and Choose}
%Considered the best for boolearn circuits ? \\
%\begin{enumerate}
%    \item Garbler ($P_1$) constructs $s$ circuits instead of 1 circuit and sends them all to evaluator ($P_2$).
%    \item $P_2$ chooses a random subset of half of those circuits and asks $P_1$ to open them i.e. $P_2$ gets access to all the keys corresponding to the input wires.
%    \item $P_2$ decrypts the entire circuit and checks that it really is a circuit computing the function.
%    \item Guarantee: Majority of the remaining circuits are correctly constructed
%    \item $P_1$ and $P_2$ compute all of the remaining garbled circuits (computing 1 not enough as it is only guaranteed that the majority are correctly constructed)
%    \item What does $P_2$ do if it obtains different outputs?
%    \begin{itemize}
%        \item What if $P_2$ aborts? $P_1$ can execute the following attack. $P_1$ constructs 1 circuit that outputs garbage if the first bit of $P_2$'s input equals 0 (otherwise, computes $f$). If $P_2$ aborts, $P_1$ knows that $P_2$'s 1st input equals 0.
%        \item What if $P_2$ outputs the majority value? Correct option. 
%    \end{itemize}
%    \item What if $P_1$ constructs a garbled circuit $G$ with 2 different sets of garbled values / keys $K$, $K'$ s.t. the keys in $K$ decrypt $G$ to the correct circuit $C$ and the keys in $K'$ decrypt $G$ to an incorrect circuit $C'$. This can be solved by having $P_1$ commit to the input keys. Later on, the computation happens on committed keys, not on arbitrary keys.
%    \item Input consistency: What if $P_2$ uses different inputs $y_1$, $y_2$, ... in different circuits to get $f(x, y_1)$, $f(x, y_2)$ etc. Similarly, what if $P_1$ uses different inputs $x_1$, $x_2$, ... in different circuits to get $f(x_1, y)$, $f(x_2, y)$ etc. $P_2$ may not detect this from the output, as the function can be customized accordingly such that it doesn't.
%    \item Cut and Choose doesn't help with selective input attack. Solution to prevent selective input attack: 
%    \begin{itemize}
%        \item Split each input bit $y$ of $P_2$ into $s$ random bits $y_1, ..., y_s$ such that $y_1 \oplus y_2 \oplus ... \oplus y_s = y$
%        \item Change circuit to first compute XOR of these bits and then the function.
%        \item Each input bit is now random (correlation between $y_1$, ..., $y_s$ and actual bit $y$ can be guessed with probability $2^{-s}$). Thus, any attack on input bits is not correlated to actual input.
%    \end{itemize}
%
%\end{enumerate}

%Consider Yao's protocol that consists of a garbler and an evaluator. Consider a malicious garbler. What can go wrong?
%\begin{enumerate}
%    \item Garbler may construct a circuit that computes a different function (Harms correctness, privacy, independence of inputs)
%    \item Garbler can carry out a selective input attack i.e. garbler can input an incorrect key for the 0-value on the 1st bit of evaluator's input and a correct key for the 1-value. If 1st bit of evaluator's input is 1, circuit decrypts correctly, else it doesn't, which reveals to garbler info on what 1st bit of evaluator's input is. This is not possible in the ideal world. (Why?)
%    \item Garbler may construct a circuit that reveals more about evaluator's input, even embeds evaluator's entire input in the output.
%\end{enumerate}
%
%\subsection{Approach of this Paper: Cut and Choose}
%Considered the best for boolearn circuits ? \\
%\begin{enumerate}
%    \item Garbler ($P_1$) constructs $s$ circuits instead of 1 circuit and sends them all to evaluator ($P_2$).
%    \item $P_2$ chooses a random subset of half of those circuits and asks $P_1$ to open them i.e. $P_2$ gets access to all the keys corresponding to the input wires.
%    \item $P_2$ decrypts the entire circuit and checks that it really is a circuit computing the function.
%    \item Guarantee: Majority of the remaining circuits are correctly constructed
%    \item $P_1$ and $P_2$ compute all of the remaining garbled circuits (computing 1 not enough as it is only guaranteed that the majority are correctly constructed)
%    \item What does $P_2$ do if it obtains different outputs?
%    \begin{itemize}
%        \item What if $P_2$ aborts? $P_1$ can execute the following attack. $P_1$ constructs 1 circuit that outputs garbage if the first bit of $P_2$'s input equals 0 (otherwise, computes $f$). If $P_2$ aborts, $P_1$ knows that $P_2$'s 1st input equals 0.
%        \item What if $P_2$ outputs the majority value? Correct option. 
%    \end{itemize}
%    \item What if $P_1$ constructs a garbled circuit $G$ with 2 different sets of garbled values / keys $K$, $K'$ s.t. the keys in $K$ decrypt $G$ to the correct circuit $C$ and the keys in $K'$ decrypt $G$ to an incorrect circuit $C'$. This can be solved by having $P_1$ commit to the input keys. Later on, the computation happens on committed keys, not on arbitrary keys.
%    \item Input consistency: What if $P_2$ uses different inputs $y_1$, $y_2$, ... in different circuits to get $f(x, y_1)$, $f(x, y_2)$ etc. Similarly, what if $P_1$ uses different inputs $x_1$, $x_2$, ... in different circuits to get $f(x_1, y)$, $f(x_2, y)$ etc. $P_2$ may not detect this from the output, as the function can be customized accordingly such that it doesn't.
%    \item Cut and Choose doesn't help with selective input attack. Solution to prevent selective input attack: 
%    \begin{itemize}
%        \item Split each input bit $y$ of $P_2$ into $s$ random bits $y_1, ..., y_s$ such that $y_1 \oplus y_2 \oplus ... \oplus y_s = y$
%        \item Change circuit to first compute XOR of these bits and then the function.
%        \item Each input bit is now random (correlation between $y_1$, ..., $y_s$ and actual bit $y$ can be guessed with probability $2^{-s}$). Thus, any attack on input bits is not correlated to actual input.
%    \end{itemize}
%
%\end{enumerate}

%\subsection{Cost of Cut and Choose}
%How many circuits are needed to make sure that the majority of the circuits sent by garbler are correct? \\

%With $s$ circuits, probability of cheating is $2^{-0.311s}$

% Bibliography
%-----------------------------------------------------------------
\begin{thebibliography}{99}

\bibitem{Lindell13} Yehuda Lindell, Bar-Ilan University \emph{Fast Cut-and-Choose Based Protocols for Malicious and Covert Adversaries}, {Crypto} (2013)

\end{thebibliography}

%\end{document}
