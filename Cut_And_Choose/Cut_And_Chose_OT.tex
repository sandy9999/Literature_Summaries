%\documentclass{article}
%\usepackage[
%    left=1.2in,
%    right=1.2in,
%    top=0.4in,
%    bottom=0.7in,
%    paperheight=11in,
%    paperwidth=8.5in
%]{geometry}

%\usepackage{layout}
\clearpage
\mytitle{Secure Two-Party Computation via Cut-and-Choose Oblivious Transfer \large \\Paper Summary}
%\renewcommand{\title}{TVA: A multi-party computation system for secure and expressive time series analytics \large \\Paper Summary}
%\author{Sandhya Saravanan\\
%  \small MSR, India\\\\
%}
%\date{\vspace{-5ex}}

%\begin{document}
%\maketitle
\setcounter{section}{0} % Restart section numbering

\section{Summary}

\subsection{Motivation}
A malicious adversary who constructs the garbled circuit may construct a garbling of a different circuit computing a different function, and this cannot be detected (due to garbling).

\subsection{Solution Methodology: Cut and Choose}
Many circuits sent, some of them are opened to check that they are correct while others are evaluated. Known to be the best methodology for Boolean circuits.

\subsection{Selling Point}
\begin{enumerate}
    \item For cheating probability of $2^{-40}$, best previous works send between 680 circuits. In this work, 132 circuits suffice.
    \item Previous works achieve cheating probability of $2^{-s/17}$. This work achieves cheating probability of $2^{-0.311s}$ where $s$ = No. of garbled circuits.
    \item Moderate increase in no. of exponentiations, reduced no. of circuits, commitments completely removed, no need to increase size of inputs.
\end{enumerate}

\subsection{Previous Works' Protocol}
\begin{enumerate}
    \item $P_1$ constructs a large no. of garbled circuits and sends them to $P_2$.
    \item $P_2$ chooses a subset of circuits which are opened and checked.
    \item If all these circuits are correct, then remaining circuits are evaluated as in Yao's protocol, and $P_2$ outputs majority output value.
\end{enumerate}

This work relies on decisional DDH assumption, while previous works relied on general assumptions.

\subsection{What if $P_2$ aborts?}
If $P_2$ receives different outputs in different circuits, they know that $P_1$ cheated. Can't they ($P_2$) abort?

\underline{Problem}: A malicious $P_1$ can construct a single incorrect circuit that computes the following function: if first bit of $P_2$'s input is 0, output random garbage, else compute correct function. If first bit of $P_2$ is 0, then they receive different output in this circuit and in the others. Else, they receive same output in all circuits. If $P_2$ aborts, $P_1$ knows that first bit of $P_2$'s input is 0.

\subsection{This Work's Protocol}
\begin{enumerate}
    \item Phase 1
    \begin{itemize}
        \item $P_1$ constructs $s$ circuits (for error $2^{-s}$). Each circuit is independently chosen as a check or evaluation circuit with prob. 1/2.
        \item If all of the circuits successfully evaluated by $P_2$ give same output $z$, then $P_2$ locally stores $z$. Else, $P_2$ stores proof that it received 2 inconsistent output values in 2 different circuits (Proof could have garbled value associated with 0 on an output wire in 1 circuit, and garbled value associated with on an output wire in another circuit).
    \end{itemize}
    \item Phase 2
    \begin{itemize}
        \item $P_1$ inputs same input $x$ as earlier, and proves it.
        \item $P_2$ inputs random values if it received single output $z$ in Phase 1, inputs proof of inconsistent values otherwise.
        \item If $P_2$'s input is a valid proof of inconsistent output values, then $P_2$ receives $P_1$'s input $x$, otherwise it receives nothing.
    \end{itemize}
    \item Output Determination
    \begin{itemize}
        \item If $P_2$ received a single output $z$ in Phase 1, then it outputs $z$ and halts. Else, if it received inconsistent values, then it received $x$ in Phase 2. $P_2$ locally computes $z = f(x, y)$ and outputs it.
    \end{itemize}
\end{enumerate}

$P_2$ doesn't provide any indication as to whether $z$ was received from Phase 1 or locally computed.

\subsection{Security Proof}
\subsubsection{Case 1}
$P_1$ is corrupted and may not construct the garbled circuits correctly. If all check circuits are correct and all evaluation circuits are incorrect, then $P_2$ may receive same incorrect output in phase 1 and will output it. This happens with prob. $2^{-s}$.\\

If all evaluation circuits are correct, correct output is obtained by $P_2$.\\

If 2 different evaluation circuits give 2 different outputs, $P_2$ obtains required "proof of cheating", learns $x$ in Phase 2, therefore locally computes and outputs correct value. $P_1$ cannot determine which case yielded output for $P_2$.

\subsubsection{Case 2}
$P_2$ is corrupted. Only way to cheat is to provide output in Phase 2 in a way that enables it to receive $x$ from $P_1$. Since $P_1$ constructs circuits correctly, $P_2$ will not obtain inconsistent outputs and cannot provide any proof.\\

Actually, single circuit is enough.

\subsubsection{Input consistency for $P_1$}
All known protocols have mechanism to ensure that $P_1$ uses same input in all computed circuits. \\

(Didn't understand the part to do with making circuit for Phase 2 small, will update after checking out Protocol)

\subsection{Primitive: Cut-and-Choose OT}
1-out-of-2 OT s.t. in some of the transfers, receiver obtains single value (like regular OT), while in others, obtains both values. \\

\underline{Use Case in Yao's GC}: Receiver obtains all garbled input values in circuits to be \underline{checked}, and only garbled input values associated with input on circuits to be \underline{evaluated}. \\

In earlier work, receiver obtains both values in exactly half of the transfers. In this work, receiver can choose at its own will in which transfers (unknown to the other) it receives 1 value, and in which it receives both. \\

\underline{Problem}: Receiver needs to prove to sender for which transfers it receives both values and for which it received only 1. \\

\underline{Solution}: Sender inputs $s$ random check values, and the receiver obtains such a value in every transfer for which it receives a single value only. (Didn't understand)
%Consider Yao's protocol that consists of a garbler and an evaluator. Consider a malicious garbler. What can go wrong?
%\begin{enumerate}
%    \item Garbler may construct a circuit that computes a different function (Harms correctness, privacy, independence of inputs)
%    \item Garbler can carry out a selective input attack i.e. garbler can input an incorrect key for the 0-value on the 1st bit of evaluator's input and a correct key for the 1-value. If 1st bit of evaluator's input is 1, circuit decrypts correctly, else it doesn't, which reveals to garbler info on what 1st bit of evaluator's input is. This is not possible in the ideal world. (Why?)
%    \item Garbler may construct a circuit that reveals more about evaluator's input, even embeds evaluator's entire input in the output.
%\end{enumerate}
%
%\subsection{Motivation}
%We need solutions to force the garbler/circuit constructor to behave honestly. Possible approaches:
%\begin{enumerate}
%    \item Prove correctness of circuit construction using ZK
%    \item LEGO: prove correctness of gate construction, and then solder gates together
%    \item Virtual MPC
%    \item From multiplication triples to arithmetic circuit construction
%    \item Cut and choose to prove correctness of Yao's circuits 
%\end{enumerate}
%
%\subsection{Approach 1: ZK Proving}
%\begin{enumerate}
%    \item Encrypt gates using asymmetric encryption with algebraic structure. Use Camenisch-Shoup based on DCR (N-residuosity): 2 exponentiations mod $N^2$
%    \item Use structure to prove in ZK that circuit is correctly constructed. E.g. used correct keys, gate has correct structure etc
%    \item Cost: $O(1)$ exponentiations/gate (Here $O(1)$ is 720), more expensive than DH exponentiations in EC group.
%    \item Scope for optimization:
%    \begin{itemize}
%        \item How to build gates so that they yield efficient proofs, and therefore, more efficient ZK protocols?
%        \item Batching of ZK protocols?
%    \end{itemize}
%\end{enumerate}
%
%\subsection{Approach 2: LEGO}
%\begin{enumerate}
%    \item Generate many encrypted gates using homomorphic (Petersen) commitments
%    \item Open half of the gates to check that they are correctly formed (Guarantees that majority of remaining gates are correct). 
%    \item Combine remaining gates in fault tolerant circuit (can tolerate some no. of incorrect gates and still correctly compute the function). Use homomorphic property to solder gates
%    \item Compute above circuit 
%    \item Size of fault tolerant circuit: $O(s|C|/\log|C|)$. $s$ is statistical security param. Error is $2^{-s}$.
%    \item Why is the size of the fault tolerant circuit so big? For every gate to be computed, we replace it with $s$ gates and take the majority result of these gates.
%    \item No. of exponentiations (DH) / gate = 32
%\end{enumerate}
%
%\subsection{Approach 4: Multiplication Tuples (Arithmetic Circuits)}
%\begin{enumerate}
%    \item Parties share their inputs like in BGW
%    \item Assume that parties have many tuples of the form [Com(x), Com(y), Com(z)] where $x = y.z$ together with additive shares $(x_1, x_2), (y_1, y_2), (z_1, z_2)$ of $x, y, z$. 
%    \item Note: Com is homomorphic
%    \begin{itemize}
%        \item Given share of Com(x), Com(y), we can compute shares of Com(x+y).
%        \item Given shares of a, x, we can compute shares of Com(a.x). 
%    \end{itemize}
%    \item Setting 
%    \begin{itemize}
%        \item Wire 1: $P_1$ and $P_2$ have additive shares $u_1$, $u_2$ of $u$.
%        \item Wire 2: $P_1$ and $P_2$ have additive shares $v_1$, $v_2$ of $v$.
%        \item Aim: Compute shares of $w=u.v$ i.e. compute $w_1$, $w_2$ s.t. $w_1 + w_2 = (u_1 + u_2).(v_1 + v_2)$
%        \item $P_1$ and $P_2$ have additive shares of Com(x), Com(y), Com(z) where $x =y.z$.
%        \item They compute shares of Com(u-y) and open. Let's call this u'
%        \item They compute shares of Com(v-z) and open. Let's call this v'
%        \item They compute shares of Com(u'.v) + Com(v'.u) + Com(x) - u'.v'
%    \end{itemize}
%\end{enumerate}
%
%\subsection{Approach of this Paper: Cut and Choose}
%Considered the best for boolearn circuits ? \\
%\begin{enumerate}
%    \item Garbler ($P_1$) constructs $s$ circuits instead of 1 circuit and sends them all to evaluator ($P_2$).
%    \item $P_2$ chooses a random subset of half of those circuits and asks $P_1$ to open them i.e. $P_2$ gets access to all the keys corresponding to the input wires.
%    \item $P_2$ decrypts the entire circuit and checks that it really is a circuit computing the function.
%    \item Guarantee: Majority of the remaining circuits are correctly constructed
%    \item $P_1$ and $P_2$ compute all of the remaining garbled circuits (computing 1 not enough as it is only guaranteed that the majority are correctly constructed)
%    \item What does $P_2$ do if it obtains different outputs?
%    \begin{itemize}
%        \item What if $P_2$ aborts? $P_1$ can execute the following attack. $P_1$ constructs 1 circuit that outputs garbage if the first bit of $P_2$'s input equals 0 (otherwise, computes $f$). If $P_2$ aborts, $P_1$ knows that $P_2$'s 1st input equals 0.
%        \item What if $P_2$ outputs the majority value? Correct option. 
%    \end{itemize}
%    \item What if $P_1$ constructs a garbled circuit $G$ with 2 different sets of garbled values / keys $K$, $K'$ s.t. the keys in $K$ decrypt $G$ to the correct circuit $C$ and the keys in $K'$ decrypt $G$ to an incorrect circuit $C'$. This can be solved by having $P_1$ commit to the input keys. Later on, the computation happens on committed keys, not on arbitrary keys.
%    \item Input consistency: What if $P_2$ uses different inputs $y_1$, $y_2$, ... in different circuits to get $f(x, y_1)$, $f(x, y_2)$ etc. Similarly, what if $P_1$ uses different inputs $x_1$, $x_2$, ... in different circuits to get $f(x_1, y)$, $f(x_2, y)$ etc. $P_2$ may not detect this from the output, as the function can be customized accordingly such that it doesn't.
%    \item Cut and Choose doesn't help with selective input attack. Solution to prevent selective input attack: 
%    \begin{itemize}
%        \item Split each input bit $y$ of $P_2$ into $s$ random bits $y_1, ..., y_s$ such that $y_1 \oplus y_2 \oplus ... \oplus y_s = y$
%        \item Change circuit to first compute XOR of these bits and then the function.
%        \item Each input bit is now random (correlation between $y_1$, ..., $y_s$ and actual bit $y$ can be guessed with probability $2^{-s}$). Thus, any attack on input bits is not correlated to actual input.
%    \end{itemize}
%
%\end{enumerate}

%\subsection{Cost of Cut and Choose}
%How many circuits are needed to make sure that the majority of the circuits sent by garbler are correct? \\

%With $s$ circuits, probability of cheating is $2^{-0.311s}$

% Bibliography
%-----------------------------------------------------------------
\begin{thebibliography}{99}

\bibitem{Lindell13} Yehuda Lindell, Bar-Ilan University \emph{Fast Cut-and-Choose Based Protocols for Malicious and Covert Adversaries}, {Crypto} (2013)

\end{thebibliography}

%\end{document}
