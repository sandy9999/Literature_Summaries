%\documentclass{article}
%\usepackage[
%    left=1.2in,
%    right=1.2in,
%    top=0.4in,
%    bottom=0.7in,
%    paperheight=11in,
%    paperwidth=8.5in
%]{geometry}

%\usepackage{layout}
\clearpage
\mytitle{New Idea}
%\renewcommand{\title}{TVA: A multi-party computation system for secure and expressive time series analytics \large \\Paper Summary}
%\author{Sandhya Saravanan\\
%  \small MSR, India\\\\
%}
%\date{\vspace{-5ex}}

%\begin{document}
%\maketitle
\setcounter{section}{0} % Restart section numbering

\section{Summary}
\subsection{Problem Statement}
\underline{Threshold Edit Distance Filtering}: We have a private query word q of length l\textsubscript{q} and a public dictionary of n words, each word of length l (We consider equal length words in the dictionary just for simplicity, the solution is applicable even otherwise). We want to find the number of words from the dictionary that are at edit distance $\tau$ from the query word. 

\underline{Application}: This is especially useful in case of a spell checker, where we want to suggest potential autocorrect options (candidate words) for a misspelled word. These candidate words are shown only if they're at a threshold edit distance from the misspelled word.
\subsection{Related Work}

Popular application of edit distance computation is in the k-Closeness problem which can be thought of as having a private query word q of length l\textsubscript{q} and a private database of n words, each word of length l and we want to find k words in the DB with the smallest edit distances from our query word q.\\

To solve this problem, a naive solution is to find the edit distance between the query word and every word in the DB, return the k words associated with the smallest distances. The communication complexity of this would be $O(nll_q)$ non-free operations and round complexity would be $O(max(l, l_q)) + r_{min}$ rounds.\\

To solve the same problem, there is a line of work where an approximate edit distance function is considered. The resulting communication complexity would then be $O(n(l + l_q))$ non-free operations and round complexity would be $O(c) + r_{min}$ rounds. For q queries, communication complexity would be $O(qn(l + l_q))$ non-free operations.\\

Now, let's consider using the approximate edit distance to solve the Threshold Edit Distance Filtering problem. A naive approach would be to compare the query word against every word in the database. This would result in a communication complexity of $O(n(l + l_q))$ non-free operations. Could we get to a communication complexity of $O(\log n(l + l_q))$ non-free operations? \\

In algorithms, we have a structure called BK Tree as a possible solution for this problem (Whether it's extensible to MPC is something we will have to check).\\

\underline{Description}: We store the dictionary in the form of a BK tree. We pick a random word from the dictionary to represent the root node r. Now, for every word we iterate (call it w), we find d = ApproxEditDistance(w, r). If there's no other node connected to the root node with edge weight d, we connect r to w with edge weight d. Otherwise, we traverse to this node via the edge with edge weight d, and repeat the process with this node as the new root node. Since the dictionary is public, none of these steps involve communication.\\

An important property of edit distance is that it obeys triangle inequality. Consider that we have 3 words, a, b and c. EditDistance(a, b) + EditDistance(b, c) $>$ EditDistance(a, c).\\

Let the edit distance between query word q and the word at the root node, r be $d$. Then, it's enough to query words from the dictionary at edit distance between $d - \tau$ and $d + \tau$. We traverse edges with weights between $d - \tau$ and $d + \tau$ (Think: DFS traversal on a tree with selectively visiting child nodes whose edge weight lie in some given range).


% Bibliography
%-----------------------------------------------------------------
%\begin{thebibliography}{99}

%\bibitem{TVA22} Muhammad Faisal, Boston University; Jerry Zhang \emph{TVA: A multi-party computation system for secure and expressive time series analytics}, {USENIX} (2023)

%\end{thebibliography}

%\end{document}
